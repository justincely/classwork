\documentclass[a4paper,11pt]{article}
\usepackage{graphicx}
\usepackage{enumerate}

\begin{document}

\begin{flushright}

\vspace{1.1cm}

{\bf\Huge Problem Set 1}

\rule{0.25\linewidth}{0.5pt}

\vspace{0.5cm}
%Put Authors
Justin Ely
\linebreak
\newline
%Put Author's affiliations
\footnotesize{605.204.81.FA15 Computer Organization\\}
\vspace{0.5cm}
% Date here below
08 September, 2015
\end{flushright}

\noindent\rule{\linewidth}{1.0pt}


\section*{1.2)}

 \begin{enumerate}[(a)]
   \item {\bf Performance via Pipelineing}
   \item {\bf Dependability via Redundancy}
   \item {\bf Performance via Predication}
   \item {\bf Hierarchy of Memories}
   \item {\bf Performance via Parallelization} 
   \item {\bf Make the common case fast}
   \item {\bf Design for Moore's law}
   \item {\bf Use Abstraction to simplify design}
 \end{enumerate}

%%%%%%%%%%%%%%%%%%%%%%%%%%%%%%%%%%%%%%%%%%%%%%%%%%%%%%%%%%

\section*{1.4)} 

\subsection*{a]}

\begin{eqnarray}
   N_{bytes} = \frac{bits}{color} * n_{colors} *  n_{pixels} * \frac{bytes}{bit} \\
   N_{bytes} = 8 * 3 * (1280 * 1024) * \frac{1}{8} \\
   N_{bytes} = 3,932,160
\end{eqnarray}

The equations specified above yield $3,932,160$ bytes for the minimum size of the frame buffer.

\subsection*{b] }

\begin{eqnarray}
   t_{sec} = \frac{\frac{bits}{color} * n_{colors} *  n_{pixels}}{ \frac{bits}{sec}} \\
   t_{sec} = \frac{8 * 3 * (1280 * 1024)}{10^{8}} \\
   t_{sec} = .315 s  
\end{eqnarray}

To calculate the time for a frame to be sent over a connection, we simply need to divide the total bits by the connection speed (also in bits/sec).  For the values given in the problem, it would take .315 seconds to transfer a frame.  

%%%%%%%%%%%%%%%%%%%%%%%%%%%%%%%%%%%%%%%%%%%%%%%%%%%%%%%%%%

\section*{1.10)}
\begin{enumerate}
  \item [Die 1] 15cm diameter, 12 cost, 84 dies, .020 defects per $cm^2$
  \item [Die 2] 20cm diameter, 15 cost, 100 dies, .031defects per $cm^2$
\end{enumerate}

\subsection*{1.10.1]}

\begin{equation}
yield = 1- \frac{area * \frac{defects}{area}}{n_{dies}}
\end{equation}

This equation provides an worst-case scenario to the yield, where each defect hits a different die and causes the maximum number of losses.  Using this equation, the yield for Die 1 is 95.8\% while Die 2 is 90.3\%.

\subsection*{1.10.2]}

\begin{equation}
cost_{die} = \frac{cost_{wafer}}{dies\_per\_wafer * yield}
\end{equation}

Using Equations 7 and 8; Die 1 has a cost of $\frac{12}{84 * .958} = .15$ per die while Die 2 has a cost of $\frac{15}{100 * .0087} = .167$ per die.

%%%%%%%%%%%%%%%%%%%%%%%%%%%%%%%%%%%%%%%%%%%%%%%%%%%%%%%%%%

\section*{1.12)}

\begin{enumerate}
  \item [P1]: 4GHz, .9 CPI, 5E9 instructions
  \item [P2]: 3GHz, .75 CPI, 1E9 instructions
\end{enumerate}

\subsection*{1.12.1]}

\begin{equation}
t_{cpu} = \frac{N_{instructions} * CPI}{f_{CPU}}
\end{equation}

For P1 equation 9 becomes $\frac{5E9 * .9}{4E9}$ or  1.125 seconds, while P2 becomes $\frac{1E9 * .75}{3E9}$ or .25 seconds.

\subsection*{1.12.2]}

The time for P1 to compute 1E9 instructions will be $\frac{1E9 * .9}{4E9}$ or .225 s.  If P2 has this long to work, it would only be able to execute $\frac{.225 * 3E9}{.75}$ = 9E8 instructions.  


\subsection*{1.12.3]}
\begin{equation}
MIPS = \frac{f_{CPU}}{CPI * 10^{6}}
\end{equation}

The MIPS for P1 is $\frac{4E9}{.9 * 10**{6}} = 4444.44$, while P2 is $\frac{3E9}{.75*10**6}$ = 4000.

\end{document}
