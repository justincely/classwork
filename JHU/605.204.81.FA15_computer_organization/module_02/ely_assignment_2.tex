\documentclass[a4paper,11pt]{article}
\usepackage{graphicx}
\usepackage{enumerate}

\begin{document}

\begin{flushright}

\vspace{1.1cm}

{\bf\Huge Problem Set 2}

\rule{0.25\linewidth}{0.5pt}

\vspace{0.5cm}
%Put Authors
Justin Ely
\linebreak
\newline
%Put Author's affiliations
\footnotesize{605.204.81.FA15 Computer Organization\\}
\vspace{0.5cm}
% Date here below
15 September, 2015
\end{flushright}

\noindent\rule{\linewidth}{1.0pt}


%%%%%%%%%%%%%%%%%%%%%%%%%%%%%%%%%%%%%%%%%%%%%%%%%%%%%%%%%%

\section*{3.1)}
$5ED4 - 07A4 = 5730$.  \\
The subtraction here was straightforward, as there were no carries.  In base 16:
\begin{verbatim}
4-4=0
D-A=3
E-7=7
5-0=5
\end{verbatim}

%%%%%%%%%%%%%%%%%%%%%%%%%%%%%%%%%%%%%%%%%%%%%%%%%%%%%%%%%%

\section*{3.2)} 

\begin{tabular}{| c | c | c | c | }
  \hline			
  5 & E & D & 4  \\
  \hline
  0101 & 1110 & 1101 & 0100  \\
  \hline  
   \hline			
  0 & 7 & A & 4  \\
  \hline
  0000 & 0111 & 1010 & 0100  \\
  \hline  
\end{tabular} \\

Using the left-most bit as the sign, this mathematical operation is subtracting a small positive number from a larger positive number.  There will be no sign change and no underflow, and the HEX value will still end up being 5730.

%%%%%%%%%%%%%%%%%%%%%%%%%%%%%%%%%%%%%%%%%%%%%%%%%%%%%%%%%%

\section*{3.3)} 

\begin{tabular}{| c | c | c | c | }
  \hline			
  5 & E & D & 4  \\
  \hline
  0101 & 1110 & 1101 & 0100  \\
  \hline  
\end{tabular} \\

Hex is attractive because it's easier for humans to read and remember then straight binary.  It is also convenient because
most memory sizes are multiples of 4, which make representation by hex straightforward.

%%%%%%%%%%%%%%%%%%%%%%%%%%%%%%%%%%%%%%%%%%%%%%%%%%%%%%%%%%

\section*{3.4)} 

$4365 - 3412 = 0753$.   \\
The first two (from right) columns are straightforward to subtract, with no borrows.  The third requires a borrow from the fourth column.

\begin{verbatim}
5-2=3
6-1=5
(initial) 3-4 -> (after carry from next) 11-4=7
(initial) 4-3 -> (after carry from previous) 3-3=0
\end{verbatim}

%%%%%%%%%%%%%%%%%%%%%%%%%%%%%%%%%%%%%%%%%%%%%%%%%%%%%%%%%%

\section*{3.5)} 

\begin{tabular}{| c | c | c | c | }
  \hline			
  4 & 3 & 6 & 5  \\
  \hline
  100 & 011 & 110 & 101  \\
  \hline  
   \hline			
  3 & 4 & 1 & 2  \\
  \hline
  011 & 100 & 001 & 010  \\
  \hline  
\end{tabular} \\


Using 2's complement to change to addition: \\

\begin{tabular}{ r   }	
   100011110101 \\
  \hline
  + 100011110110 \\
  \hline
  \hline
  = 000111101011 \\
\end{tabular} \\

which evaluates to 4753 in Octal.


%%%%%%%%%%%%%%%%%%%%%%%%%%%%%%%%%%%%%%%%%%%%%%%%%%%%%%%%%%

\section*{3.6)} 
Unsigned, 8-bit integers can hold values from 0-255 ($2^8 - 1$).  185 - 122 is 63, well within this range, and would cause neither overflow nor underflow.

%%%%%%%%%%%%%%%%%%%%%%%%%%%%%%%%%%%%%%%%%%%%%%%%%%%%%%%%%%

\section*{3.9)} 
Signed, 8-bit, integers have a range of -128 to 127.  Both 151 and 214 would exceed this maximum range, and cause overflow.  However, saturating arithmetic means to simply replace the overflow value with the max of the range.  Therefore:

$151+214=127+127=127$

%%%%%%%%%%%%%%%%%%%%%%%%%%%%%%%%%%%%%%%%%%%%%%%%%%%%%%%%%%

\section*{3.10)} 
Similar to 3.9 above, saturation of the decimal values means they would be replaced by the maximum signed value:

$151-214=127-127=0$

%%%%%%%%%%%%%%%%%%%%%%%%%%%%%%%%%%%%%%%%%%%%%%%%%%%%%%%%%%

\section*{3.11)} 
Unsigned 8-bit integers can represent values from 0-255, so this value would also saturate at the highest value.

$151+214=255$

%%%%%%%%%%%%%%%%%%%%%%%%%%%%%%%%%%%%%%%%%%%%%%%%%%%%%%%%%%

\section*{3.32)} 

%%%%%%%%%%%%%%%%%%%%%%%%%%%%%%%%%%%%%%%%%%%%%%%%%%%%%%%%%%

\section*{3.33)} 

%%%%%%%%%%%%%%%%%%%%%%%%%%%%%%%%%%%%%%%%%%%%%%%%%%%%%%%%%%

\section*{3.34)} 
No,  floating point arithmetic is not associative, and the order of evaluation can change the result.  

%%%%%%%%%%%%%%%%%%%%%%%%%%%%%%%%%%%%%%%%%%%%%%%%%%%%%%%%%%

\end{document}
