\documentclass[a4paper,11pt]{article}
\usepackage{graphicx}
\usepackage{enumerate}

\begin{document}

\begin{flushright}

\vspace{1.1cm}

{\bf\Huge Problem Set 5}

\rule{0.25\linewidth}{0.5pt}

\vspace{0.5cm}
%Put Authors
Justin Ely
\linebreak
\newline
%Put Author's affiliations
\footnotesize{605.204.81.FA15 Computer Organization\\}
\vspace{0.5cm}
% Date here below
06 October, 2015
\end{flushright}

\noindent\rule{\linewidth}{1.0pt}


%%%%%%%%%%%%%%%%%%%%%%%%%%%%%%%%%%%%%%%%%%%%%%%%%%%%%%%%%%

\section*{2.20)}

%%%%%%%%%%%%%%%%%%%%%%%%%%%%%%%%%%%%%%%%%%%%%%%%%%%%%%%%%%

\section*{2.25.1)}

The J-type instruction would be the most appropriate format.  OR I-TYPE?  Kinda like a branch?

\section*{2.25.2)}
\begin{verbatim}
    sgt, $t3, $t2, $zero
    bnq, $t3, $zero, sub
sub:
    addi, $t3, $t3, -1
    j LOOP
\end{verbatim}

%%%%%%%%%%%%%%%%%%%%%%%%%%%%%%%%%%%%%%%%%%%%%%%%%%%%%%%%%%

\section*{2.26.1)}
This MIPS loop subtracts 1 from \$t1 and adds 2 to \$s2 every iteration that \$t1 is not equal to 0.  With \$t1 initially set to
10, the loop will go through 10 iterations before finishing.  This means \$s2 will have a value of 20 after the last iteration.

\section*{2.26.2)}
A literal translation line-by-line looks like the following for each iteration:
\begin{verbatim}
if (0 < i) {
    TEMP = 1;
}
else {
    TEMP = 0'
}

if (TEMP == 0){
    break;
}

i--;
B = B+2;
\end{verbatim}

\noindent While a more concise version of the whole loop could be done with a while loop:
\begin{verbatim}
while (i > 0){
    i--;
    B = B+2;
}
\end{verbatim}

\section*{2.26.3)}
Each loop executers 5 instructions, except for the last which only executes 2 (passing by the add, subtract, and jump).  If \$t1 is initialized to N, then the total number of instructions executed is $5*(N-1) + 2$.

%%%%%%%%%%%%%%%%%%%%%%%%%%%%%%%%%%%%%%%%%%%%%%%%%%%%%%%%%%
\section*{Pseudocode: abs  \$s4, \$s1)}
The absolute value pseudocode takes 3 MIPS instructions to complete:  SLT to check if less than 0, BEQ skip over 
the SUB step if already positive, and a SUB step to correct the negative.
\begin{verbatim}
     slt, $t1, $s1, $0
     beq, $t1, $zero, skip
     sub, $s4, $0, $s1
skip:
\end{verbatim}

%%%%%%%%%%%%%%%%%%%%%%%%%%%%%%%%%%%%%%%%%%%%%%%%%%%%%%%%%%
\section*{Pseudocode: rol  \$t7, \$t3, 8)}
The roll left pseudo code takes 3 MIPS instructions: a SLL to rotate the non-discarded bits to their final place, a SRL to
move the bits discarded by the previous shift to their final location, and an OR to combine the two temporary outputs
into the final register.
\begin{verbatim}
sll $t1, $t3, 8   
srl $t2, $t3, 24
or $t7, $t2, $t1

\end{verbatim}

%%%%%%%%%%%%%%%%%%%%%%%%%%%%%%%%%%%%%%%%%%%%%%%%%%%%%%%%%%
\section*{Pseuodocode: ld  \$t5, 0(\$t8))}

\begin{verbatim}


\end{verbatim}

%%%%%%%%%%%%%%%%%%%%%%%%%%%%%%%%%%%%%%%%%%%%%%%%%%%%%%%%%%

\end{document}
