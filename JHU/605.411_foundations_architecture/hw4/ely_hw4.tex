\documentclass[a4paper,11pt]{article}
\usepackage{graphicx}
\usepackage{enumerate}

\begin{document}

\begin{flushright}

\vspace{1.1cm}

{\bf\Huge Problem Set 4}

\rule{0.25\linewidth}{0.5pt}

\vspace{0.5cm}
%Put Authors
Justin Ely
\linebreak
\newline
%Put Author's affiliations
\footnotesize{605.411 Foundations of Computer Architecture \\}
\vspace{0.5cm}
% Date here below
27 September, 2016
\end{flushright}

\noindent\rule{\linewidth}{1.0pt}

%%%%%%%%%%%%%%%%%%%%%%%%%%%%%%%%%%%%%%%%%%%%%%%%%%%%%%%%%%

\section*{1)}
The BEQ instruction is performed using {\bf c: subtraction}.

%%%%%%%%%%%%%%%%%%%%%%%%%%%%%%%%%%%%%%%%%%%%%%%%%%%%%%%%%%

\section*{2)} 
The ALU control bits 0110 signify {\bf b: subtract}.

%%%%%%%%%%%%%%%%%%%%%%%%%%%%%%%%%%%%%%%%%%%%%%%%%%%%%%%%%%

\section*{3)}
The 4-bit ALU control pattern for add is {\bf 0010}.

%%%%%%%%%%%%%%%%%%%%%%%%%%%%%%%%%%%%%%%%%%%%%%%%%%%%%%%%%%

\section*{4)}
If the zero flag is 1, then the branch will be taken.  Thus, the PC will be set to 0x08000000, incremented by 4, plus 0xAA shifted two bits 
to the left.  \\

\begin{tabular}{| l | c | c | c | c | c | c | c | c | c |}
  \hline	
  PC  & 0000 & 1000 & 0000 & 0000 & 0000 & 0000 & 0000 & 0000 \\  \hline
  +4  & 0000 & 0000 & 0000 & 0000 & 0000 & 0000 & 0000 & 0100 \\  \hline
  +(0xAA\textless \textless 2)  & 0000 & 0000 & 0000 & 0000 & 0000 & 0010 & 1010 & 1000 \\  \hline
  result & 0000 & 1000 & 0000 & 0000 & 0000 & 0010 & 1010 & 1100 \\  \hline
\end{tabular} \\

Thus, the PC will be set to {\bf 0x80002AC}

%%%%%%%%%%%%%%%%%%%%%%%%%%%%%%%%%%%%%%%%%%%%%%%%%%%%%%%%%%

\section*{5)}
From the slides in the lecture:\\

\begin{tabular}{| c | c | c |}
  \hline	
        & Memory? & Sing-extend? \\  \hline
   add & No & No \\  \hline
   addi & No & Yes \\  \hline
   not & NA & NA \\  \hline
   beq & No & Yes \\  \hline
   lw & Yes & Yes \\  \hline
   sw & Yes & Yes \\  \hline
\end{tabular} \\

\section*{5a)}
Data memory is used in lw and sw, which account for $25\% + 10\% = {\bf 35\%}$ of cycles in a single-cycle datapath.

\section*{5b)}
The sign extend circuit is needed in addi, beq, lw, and sw, which account for $20\% + 25\% + 25\% + 10\% = {\bf 80\%}$ of cycles in a 
single-cycle datapath.

%%%%%%%%%%%%%%%%%%%%%%%%%%%%%%%%%%%%%%%%%%%%%%%%%%%%%%%%%%

\section*{6a)}
\begin{tabular}{| c | c | c | c | c | c | c | c | c | c |}
  \hline	
  RegDst & ALUSrc & MemtoReg & RegWrt & MemRead & MemWrt & Branch & ALUOp1 & ALUp0 \\  \hline
  1 & 0 & 0 & 1 & 0 & 0 & 0 & 1 & 0 \\  \hline
\end{tabular} \\

\section*{6b)}
\begin{itemize}
  \item Main control
  \item ALU 
  \item ALUControl
  \item Register block
  \item PC count incrementer
\end{itemize}

\section*{6c)}
\begin{itemize}
  \item Sign extend
  \item Branch ALU
\end{itemize}

\section*{6d)}
\begin{itemize}
  \item Memory
\end{itemize}


%%%%%%%%%%%%%%%%%%%%%%%%%%%%%%%%%%%%%%%%%%%%%%%%%%%%%%%%%%

\section*{7)}
The SW instruction access the register file, the ALU, memory, and the register file again.  This totals $2 + 6 + 12 + 2 = {\bf 22ns}$.

%%%%%%%%%%%%%%%%%%%%%%%%%%%%%%%%%%%%%%%%%%%%%%%%%%%%%%%%%%

\section*{8)}
\begin{tabular}{| c | c | c | c |}
  \hline	
       101011 & 10011 & 00010 & 0000000000010100 \\  \hline
       43 & 19 & 2 & 20 \\  \hline
       sw & \$s3 & \$v0 & 20 \\  \hline
\end{tabular} \\

\section*{8a)}
The control unit's inputs are the ALUOp and the function code.  The ALUOp for the SW instruction is 00, and the function code is
ignored.

\section*{8b)}
The input values to the ALU are the contents of the two registers, the offset (immediate value), and the ALU control code for SW: 0010.

%%%%%%%%%%%%%%%%%%%%%%%%%%%%%%%%%%%%%%%%%%%%%%%%%%%%%%%%%%

\end{document}
