\documentclass[a4paper,11pt]{article}
\usepackage{graphicx}
\usepackage{enumerate}
\usepackage[usenames, dvipsnames]{color}

\begin{document}

\begin{flushright}

\vspace{1.1cm}

{\bf\Huge Problem Set 8}

\rule{0.25\linewidth}{0.5pt}

\vspace{0.5cm}
%Put Authors
Justin Ely
\linebreak
\newline
%Put Author's affiliations
\footnotesize{605.411 Foundations of Computer Architecture \\}
\vspace{0.5cm}
% Date here below
31 October, 2016
\end{flushright}

\noindent\rule{\linewidth}{1.0pt}

%%%%%%%%%%%%%%%%%%%%%%%%%%%%%%%%%%%%%%%%%%%%%%%%%%%%%%%%%%

\section*{1a)}
To select chip b, the chip selector bit will be 0, so cell 74 stays
as 74, and 0xD132 = 53554. \\

\noindent li \$t1, 74 \\
\noindent li \$t2, 53554


\section*{1b)}
To select chip a, the chip selector bit will be 1, so cell 43 becomes 
2147483691, and 0xB133 = 45363. \\

\noindent li \$t1, 2147483691 \\
\noindent li \$t2 45363

%%%%%%%%%%%%%%%%%%%%%%%%%%%%%%%%%%%%%%%%%%%%%%%%%%%%%%%%%%

\section*{2a)} 
The time for each read will be $5ns + 25ns = 30ns$.  The amount
of data transferred per read will be 64 bits = 8 bytes.  Thus the 
bandwidth will be $\frac{8 bytes}{30e-9 seconds} = 2.667e8 \frac{bytes}{second}$ or .266 Gigbytes per second.


\section*{2b)}
LW retrieves 1 word, which is 32 bits or 4 bytes.  Thus, LW would need only
1 read which would take a minimum of 30ns.  The LW, in practice, would take more 
time depending on the architecture of the pipeline and the time required
for the other 4 stages. 

%%%%%%%%%%%%%%%%%%%%%%%%%%%%%%%%%%%%%%%%%%%%%%%%%%%%%%%%%%

\section*{3)}
  
\begin{itemize}
  \item a) 1
  \item b) 1
  \item c) 0
  \item d) 0
\end{itemize}

%%%%%%%%%%%%%%%%%%%%%%%%%%%%%%%%%%%%%%%%%%%%%%%%%%%%%%%%%%

\section*{4a)}
{\bf Presuming you mean problem 3 instead of problem 1:}\\

\noindent This is best characterized by a latch as the output changes in
response to the input, not in response to a clock. 


\section*{4b)}
This is best characterized by S-R, as the output state can be set or reset
based on which input is triggered, and remembers that state when the inputs
are turned off.  

%%%%%%%%%%%%%%%%%%%%%%%%%%%%%%%%%%%%%%%%%%%%%%%%%%%%%%%%%%

\section*{5Ia)}
$15e9 = \frac{4 \times N}{10e-9} = 37.5$.  Thus, 38 memory banks would be
required to provide 15 billion bytes per second.

\section*{5Ib)}
22 bits are required to address the bytes on a given chip, and 4 bits are 
required to select the chip, so the total number of bits is 26.

%%%%%%%%%%%%%%%%%%%%%%%%%%%%%%%%%%%%%%%%%%%%%%%%%%%%%%%%%%

\section*{5IIa)}
Longest stage = 20ns, so clock rate = $\frac{1}{20e-9} = .05GHz$.

\section*{5IIb)}

Width = word size = 32 bits = 4 bytes.  Depth = $\frac{2^30}{4} = 268435456 = 2^{28}$ bytes.  

\section*{5IIc)}
This pipeline would need 1 new
instruction every 20 ns, for a bandwidth of $\frac{4}{20e-9} = 2e8 \frac{bytes}{second}$.  However, DDRAM transfers twice the data with each read, so the bandwith could be halved to 1e8 $\frac{bytes}{second}$.

\section*{5IId)}
$128e6 \frac{bytes}{second} = \frac{width}{cycle time} = \frac{4 \frac{bytes}{cycle}}{cycle time}$.  Thus the cycle time is 3.125e-8 seconds/cycle.

%%%%%%%%%%%%%%%%%%%%%%%%%%%%%%%%%%%%%%%%%%%%%%%%%%%%%%%%%%

\section*{6)}
\begin{tabular}{| c | c | c |}
  \hline	
  	CMD & Big-Endian & Little-Endian \\ \hline \hline
	lw \$t0,0(\$t1) & 0x89875543 & 0x43558789 \\ \hline
	lh \$t0,0(\$t1) & 0x00008987 & 0x87890000 \\ \hline
	lb \$t0,3(\$t1) & 0x00000043 & 0x43000000\\ \hline
	lh \$t0,2(\$t1) & 0x00005543 & 0x43550000 \\ \hline
\end{tabular} \\



%%%%%%%%%%%%%%%%%%%%%%%%%%%%%%%%%%%%%%%%%%%%%%%%%%%%%%%%%%

\end{document}
