\documentclass[a4paper,11pt]{article}
\usepackage{graphicx}
\usepackage{enumerate}

\begin{document}

\begin{flushright}

\vspace{1.1cm}

{\bf\Huge Problem Set 2}

\rule{0.25\linewidth}{0.5pt}

\vspace{0.5cm}
%Put Authors
Justin Ely
\linebreak
\newline
%Put Author's affiliations
\footnotesize{615.202.81.FA15 Data Structures \\}
\vspace{0.5cm}
% Date here below
15 September, 2015
\end{flushright}

\noindent\rule{\linewidth}{1.0pt}

%%%%%%%%%%%%%%%%%%%%%%%%%%%%%%%%%%%%%%%%%%%%%%%%%%%%%%%%%%

\section*{1a)}
Set the variable i equal to the second element from the top of the stack. Leave the stack unchanged. 
\begin{verbatim}
s = Stack()
a = s.pop()
b = s.pop()
i = b

s.push(b)
s.push(a)
\end{verbatim}

\section*{1b)}
Given an integer n, set the variable i equal to the nth element from the top of the stack.  Leave the stack without its top n elements
\begin{verbatim}
s = Stack()

j = 1
while j < n:
    s.pop()
i = s.pop()
\end{verbatim}

%%%%%%%%%%%%%%%%%%%%%%%%%%%%%%%%%%%%%%%%%%%%%%%%%%%%%%%%%%

\section*{2a)}
Set the variable i equal to the bottom element of the stack. Leave the stack unchanged. 
\begin{verbatim}
s1 = stack()
s2 = stack()

while not s1.empty():
    s2.push(s1.pop())
i = s2.peek()

while not s2.empty():
    s1.push(s2.pop())
    
\end{verbatim}

\section*{2b)}
Set the variable i equal to the third element from the bottom of the stack..
\begin{verbatim}

s1 = stack()
s2 = stack()

while not s1.empty():
    s2.push(s1.pop())

s2.pop()
s2.pop()
i = s2.pop()

\end{verbatim}

%%%%%%%%%%%%%%%%%%%%%%%%%%%%%%%%%%%%%%%%%%%%%%%%%%%%%%%%%%

\section*{3a)} 
\begin{tabular}{ c || l }
  \hline			
  Iteration & Stack \\
  \hline
  0 & \{      \\
  1 & \{ [    \\
  2 & \{    \\
  3 & \{ [    \\
  4 & \{ [ (   \\
  5 & \{ [    \\
  6 & \{    \\
  \hline  
\end{tabular} \\

%%%%%%%%%%%%%%%%%%%%%%%%%%%%%%%%%%%%%%%%%%%%%%%%%%%%%%%%%%

\section*{3b)} 
\begin{tabular}{ c || l }
  \hline			
  Iteration & Stack \\
  \hline
  0 & (           \\
  1 & ( (         \\
  2 & (           \\
  3 & ( \{        \\
  4 & ( \{ (      \\
  5 & ( \{ (      \\
  6 & ( \{ ( [    \\
  7 & ( \{ ( [    \\
  8 & ( \{ (      \\
  9 & ( \{        \\
  10 & (         \\
  11 &           \\
  \hline  
\end{tabular} \\

%%%%%%%%%%%%%%%%%%%%%%%%%%%%%%%%%%%%%%%%%%%%%%%%%%%%%%%%%%

\section*{4)} 
\begin{verbatim}

def check_mirror(string):
    s = Stack()
    second_half = False
    
    for letter in string:
        if not second_half:
            s.push(letter)
        
        if letter == 'c':
            s.pop()
            second_half = True
           
        if s.empty():
            return False 
        if letter == s.pop():
            return False
            
    return True

\end{verbatim}

%%%%%%%%%%%%%%%%%%%%%%%%%%%%%%%%%%%%%%%%%%%%%%%%%%%%%%%%%%

\section*{5)}
\begin{verbatim}
# using function definition from problem 4

def pattern_match(string):
    for letter in string:
        s.push(letter)
    
        if letter == 'D':
            s.pop()
        
            if not check_mirror():


\end{verbatim}

%%%%%%%%%%%%%%%%%%%%%%%%%%%%%%%%%%%%%%%%%%%%%%%%%%%%%%%%%%

\section*{6)}


%%%%%%%%%%%%%%%%%%%%%%%%%%%%%%%%%%%%%%%%%%%%%%%%%%%%%%%%%%

\section*{7)}


%%%%%%%%%%%%%%%%%%%%%%%%%%%%%%%%%%%%%%%%%%%%%%%%%%%%%%%%%%

\section*{8)}


%%%%%%%%%%%%%%%%%%%%%%%%%%%%%%%%%%%%%%%%%%%%%%%%%%%%%%%%%%

\section*{9)}
\subsection*{9a]}
Prefix: \\
Postfix:  A B + C \$ D E - F + * G -
\subsection*{9b]}
Prefix:  \\
Postfix: A B C - D E - * F + G / \$ H J - +

%%%%%%%%%%%%%%%%%%%%%%%%%%%%%%%%%%%%%%%%%%%%%%%%%%%%%%%%%%

\section*{10)}
\subsection*{10a]}
\subsection*{10b]}
\subsection*{10c]}
$(A - B + C) ^ (D + E - F)$
\subsection*{10d]}

%%%%%%%%%%%%%%%%%%%%%%%%%%%%%%%%%%%%%%%%%%%%%%%%%%%%%%%%%%

\section*{11)}
\subsection*{11a)} 
$((A+B)-C) - ((B+A)^C) = 0$
\subsection*{11b)}
$(A x (B+C)) x (C+(B-A)) = 20$


%%%%%%%%%%%%%%%%%%%%%%%%%%%%%%%%%%%%%%%%%%%%%%%%%%%%%%%%%%

\section*{12)}


%%%%%%%%%%%%%%%%%%%%%%%%%%%%%%%%%%%%%%%%%%%%%%%%%%%%%%%%%%

\end{document}
