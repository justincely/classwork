\documentclass[a4paper,11pt]{article}
\usepackage{graphicx}
\usepackage{enumerate}
\usepackage{amsmath}
\usepackage{mathtools}
\DeclarePairedDelimiter{\ceil}{\lceil}{\rceil}

\usepackage{listings}
\usepackage {tikz}
\usetikzlibrary {positioning}
%\usepackage {xcolor}
\definecolor {processblue}{cmyk}{0.96,0,0,0}

\begin{document}

\begin{flushright}

\vspace{1.1cm}

{\bf\Huge Problem Set 3}

\rule{0.25\linewidth}{0.5pt}

\vspace{0.5cm}
%Put Authors
Justin Ely
\linebreak
\newline
%Put Author's affiliations
\footnotesize{Foundations of Algorithms \\}
% Date here below
July 5, 2016 \\
612-240-0924
\end{flushright}

\noindent\rule{\linewidth}{1.0pt}

%%%%%%%%%%%%%%%%%%%%%%%%%%%%%%%%%%%%%%%%%%%%%%%%%%%%%%%%%%
\section*{1a)}
With weights of $[5, 9, 10, 2]$ and $k=10$, the number of 
trucks used will be 4, where the minimum possible would be 3
given a more efficient algorithm.

\section*{1b)}
After playing with this for a while, and listening to office hours,
I still wasn't able to come up with an adequate solution.  The office
hours recording referenced variables that were seemingly undefined, and 
some of the assumptions made didn't seem to fit in the problem definition.

Nevertheless, I tried a bunch of scenarioes and always found the best and
worst cases to be within a factor of 2 of each-other, so it does seem to be
the correct result.

%%%%%%%%%%%%%%%%%%%%%%%%%%%%%%%%%%%%%%%%%%%%%%%%%%%%%%%%%%

\section*{2a)}
The probability of any new node to link to a specific existing node
is $P_l = \frac{1}{k-1}$.  With a node $v_j$, where only nodes added
after j can possibly have linked to it, the expected number of linked
nodes is going to be the sum of the probabilites of each new node:
\begin{eqnarray}
N_l &=& \sum_{k=j+1}^{n} \frac{1}{k-1}\\
&=& \sum_{k=j}^{n-1} \frac{1}{k} \\
&=& \int_{0}^{n-1} \frac{1}{k} - \int_{0}^{j} \frac{1}{k} \\
&=& ln(n-1) - ln(j)\\
&=& ln(\frac{n-1}{j}) \\
&=& \theta(ln(\frac{n}{j}))
\end{eqnarray}

\section*{2b)}
The probability of each new node to not link to a specific existing
node is $P = 1-\frac{1}{k-1}$.  Thus, the expectation value is given by:

\begin{eqnarray}
E &=& \prod_{k=j+1}^{n} (1-\frac{1}{k-1}) \\
&=& \frac{j-1}{j} \frac{j}{j+1} \frac{j+1}{j+2} ... \frac{n-2}{n-1} \\
&=& \frac{j-1}{n-1}
\end{eqnarray}

Thus the total number of unlinked nodes is:

\begin{eqnarray}
N &=& \sum_{j=1}^{n} \frac{j-1}{n-1} \\
&=& \frac{1}{n} \sum_{j=1}^{n} j-1 \\
&=& \frac{1}{n-1} \frac{n(n-1)}{2} \\
&=& \frac{n}{2} 
\end{eqnarray}

%%%%%%%%%%%%%%%%%%%%%%%%%%%%%%%%%%%%%%%%%%%%%%%%%%%%%%%%%%%

\section*{3a)}
Pop operations run in $O(1)$ time, and in worst case MultiPopA, MultiPopB,
and Transfer will need to exhaust the full stack.  This gives worst case
running times of:

\begin{itemize}
  \item MultiPopA: $O(n)$
  \item MultiPopB: $O(m)$
  \item Transfer: $O(n)$
\end{itemize}

\section*{3b)}
Define $\phi(n, m) = n + m$ and $\hat{c_{i}} = \phi(D_{i+1}) - \phi(D_{i})$, thus:

\subsection*{PushA}
\begin{eqnarray}
\hat{c_{i}} &=& c_{i} + \phi(D_{i+1}) - \phi(D_{i}) \\
&=& 1 + (n+1) + m - n + m \\ 
&=& 1 + 1 \\
&=& 2
\end{eqnarray}

\subsection*{PushB}
\begin{eqnarray}
\hat{c_{i}} &=& c_{i} + \phi(D_{i+1}) - \phi(D_{i}) \\
&=& 1 + n + (m+1) - n + m \\ 
&=& 1 + 1 \\
&=& 2
\end{eqnarray}

\subsection*{MultiPopA}
\begin{eqnarray}
\hat{c_{i}} &=& c_{i} + \phi(D_{i+1}) - \phi(D_{i}) \\
&=& k + (n-k) + m - n + m \\ 
&=& k - k \\
&=& 0
\end{eqnarray}

\subsection*{MultiPopB}
\begin{eqnarray}
\hat{c_{i}} &=& c_{i} + \phi(D_{i+1}) - \phi(D_{i}) \\
&=& k + n + (m-k) - n + m \\ 
&=& k - k \\
&=& 0
\end{eqnarray}

\subsection*{Transfer}
\begin{eqnarray}
\hat{c_{i}} &=& c_{i} + \phi(D_{i+1}) - \phi(D_{i}) \\
&=& 2k + (n-k) + (m+k) - n + m \\ 
&=& 2k + 0 \\
&=& 2k
\end{eqnarray}

We see from the amortized cost of each operation that they each run in constant time: $O(1)$.  

\end{document}